---
title: "Privacy After Death: On the Culture of the Immortal Right"
author: Adam Fiedler
bibliography: the-immortal-right
published: 2019-03-18
mathjax: true
---

\documentclass[a4paper]{article}
\usepackage[top=1in, bottom=1.25in, left=1.5in, right=1.5in]{geometry}
\usepackage{hyperref}
\usepackage{mathtools}

\hypersetup{%
    colorlinks=true,
    urlcolor=blue,
    citecolor=black}

\title{
    Privacy After Death:\\
    On the Culture of the Immortal Right
}
\author{
    Adam Fiedler\\
    \small
    Masaryk University, Brno, Czech Republic\\
    \small
    \texttt{\href{mailto:fiedlr@mail.muni.cz}{fiedlr@mail.muni.cz}}
}
\date{}

\begin{document}
\maketitle

The right to privacy is rooted to varying degrees within human culture. It is even mentioned in the \textit{Universal Declaration of Human Rights} \cite{udrights} and over the years, many countries have set up some form of legal framework covering one's modern needs in this respect \cite{consumerreport}. One of the latest attempts here in the EU having been called the \textit{General Data Protection Regulation} (or GDPR in short) among other things tries to respond to new societal developments associated with big data \cite{louveaux}.

Privacy has been an important topic heavily discussed again with the rise of the Internet. Besides turning to our own opinions and feelings about how it relates to freedom, one can find reasonable arguments calling for its importance \cite{harbinja2}. 
However, possibilities in terms of digital storage grow so much that detailed personal data acquired during an \emph{entire lifetime} of a today's individual, collectively termed \textit{digital remains} \cite{buitelaar}, might be stored in the near future. 

That naturally opens the question of why or whether at all digital remains should be kept private. Along with introducing the reader with the concept of \textit{postmortem privacy}, the following essay would like to explore new arguments against this immortal right and show that it is currently more of a \emph{cultural norm} than a well grounded principle such as its antemortem counterpart.

\medskip
By postmortem privacy is meant the right of a person to preserve and control what becomes of his reputation and dignity after death \cite{harbinja2, buitelaar}. When we want to see just how much ingrown it is into the Western culture, we can begin, for example, by analyzing the language of this very essay. Consulting the \textit{Oxford Dictionaries}, which should at least in theory reflect the usage of English, they refer to a presumably well-known phrase ``take the (or one's etc.) secret to the grave'' \cite{oxford}. Analogous phrases occur in other popular European languages, notably German \cite{duden}, French \cite{larousse} or Polish \cite{wsjp}.

References to this phenomenon have also appeared in the art world, in lives of the fictional characters as well as of artists themselves. Take Franz Kafka, who is now widely recognized as a major literary figure of the 20th century. 
Kafka's fame started to rise mainly after many of his long hidden works had posthumously been published \emph{against} his will by a close friend named Max Brod, who went through his notes and put them together \cite{kafka, brod}.

There is Borges's short story \textit{The Mirror and the Mask}, which introduces two characters: a king and a poet. In the end, they both get to know ``the sin of having known Beauty, which is a gift forbidden mankind'', while right afterwards the poet kills himself, the king becomes a beggar and ``he has never spoken the poem again'' \cite{borges}. 

Gustave Doré's drawing \textit{Bluebeard, his wife, and the keys} \cite{dore} depicting the French legend \textit{Barbe bleue}, which tells the tale of a serial widower and his new wife. It is given a whole new interpretation in the Béla Bartók's opera \textit{Bluebeard's castle}. 
When the wife is welcomed to his castle, she learns of seven mysterious doors hiding the widower's darkest secrets. 
Initially he does not want to let her behind them, although with time, after she assures him of her devotedness, he eventually shows her each and every one of the rooms, while the last leads to her own death, leaving Bluebeard alone again with his secrets \cite{keiser}.

It is not only the historical art works, the same theme might be more relevant today than ever before. Some of today's artists and entertainment makers find its therapeutic effects \cite{sophie-calle}, while some ponder over how technology is gradually paving way to disturb the immortal right in general. 
Although obviously we do not put every piece of information we possess online, it is not little \cite{marr}. Subsequently, much can be derived from this information if someone gives enough effort to finding out \cite{like, datascience}.

Imagine the scenario that after you pass away, your life worth's of data is fed into a neural network standing behind a chat bot. Of course, no program as of now has passed a lengthy \textit{Turing test} yet (in fact, there is a public wager on when this should happen \cite{longbet}), but there are some results on the practical side of the field. 
One chatbot is reportedly ``built using the vast social content of the Internet'' and to date, more than 5,000 users have had over an hour-long conversation with it \cite{microsoft}. Bots were even being taught how to negotiate \cite{fbbots}. 
Pop culture is not late to the stage as usual in showing potential controversies. Once callibrated, an android bot fed with your data after your demise may in the future respond very much like you and puzzle people you knew \cite{blackmirror}.

\medskip
The preceding should suffice to illustrate that at least Westerners are interested in privacy and do care about how others think of them not only while alive, but after their death. It is no accident that from childhood we are told \textit{say nothing but good of the dead} \cite{collins}. 
We could say it has moral grounds (e.g. the dead cannot defend themselves), in the not so distant a past we would even say noble or sacred grounds as if the deceased deserved ``admiration for one who has accomplished a very difficult feat'' \cite{freud}, but it might also show that human \emph{self-image} is everlasting.

Even though a chatbot would probably not be able to come up with very original sentences with current technology, it does not necessarily need to. 
There is imaginably a market with all sorts of relieving services for weeping relatives, all it needs is one relative who gets access to your account and who is willing to give it up for such kind of a service. 
Based on how neural networks work, no one can guarantee that the chat bot will not use anything controversial that you did not know or forgot about being saved. There are certainly raising concerns about this risk \cite{angstrom}.

What would be other reasons for using the immortal right other than self-image? Why think it reasonable people should care about personal things to stay personal forever? Naturally, in the view of the dying individual, any such reason must be based on the presumption contrary to the \textit{simulation hypothesis}: \cite{bostrom} the world around you is objective, is no mere simulation, which you were arbitrarily put into, and continues without you. This also agrees with the religious perspective meaning that you are responsible for your deeds.
Therefore it should matter what state of affairs you leave behind and burying information might have an influence on prolonging the status quo, insofar as only you have and can get access to such information.

The most easily imaginable scenario why keep a secret forever other than for your own dignity is to protect another's dignity or happiness such as your descendant's. Yet supposing this another person is still alive, then related subset of data is or should \emph{already} be protected under standard regulations, no matter it came from a deceased person's dataset. Thus it only becomes a technical problem of separating the sensitive data related to the still living, in the worst case it can be defended \emph{ad hoc}. If this person is not alive, it is the same argument: withholding a fact about a person no longer alive cannot do much but keep him or her a certain image in the society.

Avoiding vague and ambivalent principles of dignity, we would have to go to such lengths as imagining the person knew classified or potentially dangerous information to think otherwise. But again, it is very likely these cases are already protected by other laws anyway, therefore a reduction to postmortem privacy rights should not apply to them.

Due to the very definition of postmortem privacy, the studied pioneer advocates have not come up with many alternative arguments in its defense. They are based on assumptions ``person's immaterial interests may survive his death'' \cite{buitelaar}, finding new definitions of dignity, indirect causes of defamation, attempts at attributing subsisting subjects (i.e. the digital persona) a legal agent status, even theorizing about harm to these subsisting subjects and finding analogies between the living and the dead \cite{buitelaar, harbinja2}. Their arguments are not very convicing in that they are either based on doubtful assumptions or wrong conclusions. To illustrate, we shall mention just the following three, notwithstanding the obvious possibility there will always be arguments left out.

\begin{quote}
    ``Arguably, it is a plausible analogy that, since knowledge is not a necessary condition for harm in the antemorten stage, it is not a necessary condition either in the post-mortem stage.'' \cite[p. 133]{buitelaar}
\end{quote}
    
It is true that knowledge is not a necessary condition for harm, in other words, an individual can be harmed unconsciously. 
The analogy of unconscious harm to the post-mortem harm is false, however, because post-mortem harm is an oxymoron: there is no harm without effects to the harmed, and effects cannot ever be perceived by the deceased unless we would like to assert that the dead have perceptions. The \textit{no-effect injury} \cite{winter} mentioned by Winter does not apply here because changes to reputation of a living person still \emph{have a potential} to be known by the person during lifetime and that is why they matter.

Another analogy is made by Harbinja who opposes critics of postmortem privacy by reducing postmortem privacy claims to postmortem \textit{testation} claims. 

\begin{quote}
    ``Following a similar line of arguments, the deceased should not be interested in deciding what happens to their property on death as they would not be present to be harmed by the allocation.'' \cite[p. 32]{harbinja2}
\end{quote}

Allocating your property in a testament always \emph{relates} solely to living people (i.e. you, while you are alive, and the inheritors, after you die), postmortem privacy does not, what differentiates the two and makes it absurd to compare them. These two proponents of postmortem privacy speak of interests and harm where there can hardly be one. Let us look at one more claim by Buitelaar about self-interest.

\begin{quote}
``Obviously, self-centered interests provide the highest degree of fulfillment, satisfaction and pleasure when one is able to achieve them in his life-time... 
In case these interests are thwarted or even made senseless after one's death... it can be said that this event is responsible for harming the ante-mortem person.'' \cite[p. 133]{buitelaar}
\end{quote}

\noindent What possible interests could those be if they are not related to a living person? Buitelaar contradicts his own words when he paraphrases Winter in that one can only have interest if one has capacity for being interested \cite{buitelaar}. The deceased cannot strive for achievements, therefore their image cannot matter to them anymore, so there is no reason why it should not be factual. If there is anything disputable they have intentionally hidden, should they still be honored in history?

Fame is arguably the only pure postmortem self-interest a person can have, self-interest meaning so that it only and solely concerns the individual and does not transfer to his or her descendants, for instance. It could be agreed with Buitelaar that posthumous defamation is wrong \cite{buitelaar}, yet defamation can only be based on \emph{false statements} \cite{leroy} and thus your data cannot libel you as long as its authenticity is \emph{verifiable}. Moreover, fame does not recognize morality. Besides all the infamous people of history known for monstrosities, true facts about one's life will not remove credit even for any reputable work one has done. In fact, there are several counterexamples. 

Caravaggio was a killer \cite{caravaggio}, yet he is still celebrated as an innovative painter. 
Jean-Jacques Rousseau is widely known as an influential writer, who among other things wrote an educational treatise \textit{Emile, or On Education} analyzing also how to raise children properly, whereas according to his autobiography, he left his own children as foundlings \cite{rousseau}. 
Von Braun is famous to have been a leading rocket engineer who, nevertheless, had first openly collaborated with the Nazis and offically applied for membership in the Nazi Party \cite{vonbraun}. 
Despite owning a large number of slaves who worked his plantations, Thomas Jefferson continues to be generally praised for his public achievements and regarded as a leading spokesman for democracy \cite{jefferson}. It should be noted that the list could continue so well it could be even a good theme for a book.

The interested public, on the other hand, cannot but benefit from knowing facts about a person's life rather than storytelling from relatives and friends, which are potentially biased. Let us consider a stereotypical example. 
If we better understood what drove Albert Einstein to discovering the \textit{theory of relativity}, numerous other scientists could take inspiration and find similarities in their own paths. 

Much of Einstein's material is indeed published today \cite{einstein-notes}. To name one thing, we do know he was deeply influenced by Ernst Mach's \textit{relativism} even though he had initially \emph{objected} to it \cite{einstein-inspiration}. A good friend of Einstein's during his student years, an Italian/Swiss engineer named Michele Besso, is often credited with introducing him to Mach's work \cite{einstein-inspiration}. Actually in Einstein's own words, Besso helped him with some challenges he faced when dealing with special relativity \cite{einstein}. 
No matter his importance to Einstein, it is difficult to find information on Besso other than in Einstein's biographies, not to menion that he and Mach must have been hardly a fraction of the physicist's sources of intellectual and creative inspiration. 

Social media do not only record what we choose to save, they track our behavior when interacting with them. This can seem certainly annoying and disagreeable to us as the targets of marketing. 
On the other hand, had Einstein lived in the era of Internet and used a social network, no matter how often as long as it did not interfere with his results, what could have been derived from such data?

We should not have had our expectations high, it could have been just a collection of silly cat memes. When set up right though, social networks can arguably be an interesting source of information partially fit even for one's intellectual needs. Assuming Einstein would have also used it as a source of potential inspirations, his interactions as well as the sources themselves would have been unimaginably enriching resources.

The assumption it would not have interfered with his results is a dubious one because he would have probably been a different person. That does not say anything about present and future notable people who (will) use social media every day. 
The biggest player in social media, \textit{Facebook}, is reportedly used by more than two billion people each day, almost a third of the estimated world's population \cite{facebook, wp}. Hence, besides the commonly watched celebrities' accounts, there is a good chance it is used by a future \emph{Nobel~Prize} winner.

Social media can be filled with boast, however. That does not apply to private devices such as smartphones. Their sales have increased significantly since the beginning of this decade \cite{statista2}. We tend to have them as everyday assistants, calendars, to-do lists, diaries with multimedia inputs (taking note of things they were intrigued by that day)...  The options are manifold. 
Whatever the reason, some leave location services turned on to track the phone's location (and hence its owner's) most of the time and happily synchronize the phone's data with the cloud, making the data  available to third-parties for an indefinite amount of time. 

It does not matter if they lose or stop using the phone, the data is so extensive it can tell much about their traits and habits during the period they used the phone, while the rest can often be inferred \cite{quattrone}. If publicly available, it could take future historians' research to a different level because details on how a personality is formed might be a part of the missing picture in answering questions about how it then leads people to their actions. 
Just maybe it could even be the key in moving on with the Descartes's dream to find the universal principle people use to solve problems \cite{polya}, knowing more about thought processes behind great discoveries, inventions and creations.

Having this kind of personal data at hand, a branch of future historical research could then largely consist of its verification, analysis and transformation into a usable form, which would be nontrivial.
It could contain anomalies like collected subsets of data unintentionally created by other people than the owner or various forms of irrelevant data needed to be removed first as a needle from a haystack before coming to any conclusions.

We can assume after a lifelong journey of collection, the set would not be small. Consider just storing your GPS location each second of your  lifetime, that is three floating-point numbers (latitude, longitude, elevation). 
For an average life expectancy of 70 years that would count up to roughly at least 26 gigabytes \cite{calc}. Needless to say possible amount estimates about stored multimedia and interactive data from new types of devices such as \emph{wearables}, tracking the owner consistently without need of the owner's participation.

Using the same context for a summary, say a future laureate-to-be picks up a known borrowed book at the local library, spends the afternoon there, making notes of favorite passages into an electronic notes application and his or her \emph{heart rate} jumps at a certain time during the stay, leading to another book rental and actions associated with the topic of interest. 

A historian might retrospectively guess how the topic was formed better than just from traditional sources. Professionals might follow the path in learning about the topic and discovering dead ends quicker than by checking for themselves. 
The same could apply to people of non-scientific backgrounds such as politicians, artists, players and businesspeople alike. Not to mention the social aspect behind achievements, which is sometimes neglected in biographies because there is little associated information available.

We can dare to go even further as to hypothesize that people would choose their actions more carefully if they knew some actions would eventually get under scrutiny. Such a hypothesis without doubt sounds totalitarian, nonetheless by definition it cannot be so because it does not emplace any direct duties or restrictions upon any individual. With its possible effects, it rather resembles a stronger version of the \textit{Judgment Day}, as faith in the objective world is presumably stronger than in any god. 

Of course, it is entirely possible that people would opt not to save anything important in the digital storage or remove it in time before their demise if postmortem privacy would officially be rejected. That is and should be a protected right as well as privacy, autonomy, property and other important rights are. But unless they choose to, the above mentioned arguments hold.

The problem is that currently, regardless of our opinion, postmortem privacy is in uncharted waters as a matter of law, especially in the common law countries where \textit{actio personalis moritur cum persona}, meaning personal causes of action die with the person, holds quite strong \cite{harbinja2}. 

In the USA, the Privacy Act is quite clear that it does not apply to the deceased, although ``court precedents have shown that the privacy concerns of surviving family members also weigh''. They have no laws that specifically cover access to electronic accounts of the deceased \cite{grabianowski}.

On the contrary, ``this protection is, however, more prominent and encompasing in civil law countries, aiming to protect the values such as autonomy, dignity and reputation, especially of the creators'' \cite{harbinja2}. In fact in Germany, a person's dignity is a tenet of their Constitutional law \cite{buitelaar}. These cases are hence applicable to the postmortem stage as well, making them in those countries a deep matter of law inquiry. Similarities occur in a variety of European countries, such as Bulgaria or Estonia \cite{harbinja}. 

One of the reasons for these legal differences might be that they seem rather arbitrary. Research on this topic is only emerging \cite{harbinja2}. This would also agree with what we have proposed, in short that postmortem privacy right now is primarily a paradigm in the Western society, yet to be cognizant of its practical implications stemming from technological development.
It is thus very hard to predict where it is headed as a topic of legal, scientific and philosophical discourse and on what grounds will such discourse be based. One possible lead out of the countless many, it is to be hoped, has been proposed by this essay.

\medskip
What has been put forward might as well all be a mere speculation, but it tries to shed light on potential advantages in considering postmortem data as a \emph{public domain} versus the mentioned obscure disadvantages, which are questionable even in theory. We made a brief demonstration of how postmortem privacy is a part of the Western culture and questioned its supposed relevance. Afterwards, several examples and ideas have actually been proposed against it while leaving the reader food for thought about what benefits could bear a world with an intentional renouncement of such a concept.

Let us not fool ourselves. It was in no way a rigorous law analysis and that was never its aim. On the contrary, we have just shown how in essence a cultural norm can indirectly hinder the development of a potentially revolutionary new discipline. 
Big data is so focused on targetting new customers in swarms counting on their current trends that research on what can be inferred from and how to cope with lifelong datasets of particular people is believably  ripe. 

Of course, there might be problems on the practical side of things, one of them being the still little public recognition of this topic. Much of the discussion was thus presented with ifs. However, only after we understand it better in theory, we can make better decisions about putting the principles into practice. Postmortem privacy stands on an edge between individual and public interest. In the case it makes, it might as well be tilted towards the latter.

\end{document}
